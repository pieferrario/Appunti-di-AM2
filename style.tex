\usepackage[T1]{fontenc}
\usepackage[utf8]{inputenc}
\usepackage[italian]{babel}
\usepackage{amsmath, amssymb, amsthm}
\usepackage{geometry}
\usepackage{titling}
\usepackage{enumitem}
\usepackage{pdfpages}
\usepackage{graphicx, float}
\usepackage{subcaption}
\usepackage{caption}
\usepackage{hyperref}
\usepackage{mdframed} % Pacchetto per le cornici
\usepackage{xcolor}   % Pacchetto per la gestione dei colori

% Colore per il background delle cornici
\definecolor{grey245}{RGB}{245,245,245}

% Teoremi e definizioni
\theoremstyle{plain}
\newtheorem{theorem}{\textsc{Teorema}}[chapter] 
\newtheorem{lemma}[theorem]{\textsc{Lemma}}
\newtheorem{corollary}[theorem]{\textsc{Corollario}}
\newtheorem{proposition}[theorem]{\textsc{Proposizione}}

\theoremstyle{definition}
\newtheorem{definition}{\textsc{Definizione}}[chapter]  
\newtheorem{example}{\textsc{Esempio}}[chapter] 


%Cornice teorema
\newmdenv[skipabove=5pt,
skipbelow=5pt,
rightline=false,
leftline=true,
topline=false,
bottomline=false,
linecolor=purple,
backgroundcolor=purple!0,
innerleftmargin=5pt,
innerrightmargin=5pt,
innertopmargin=0pt,
innerbottommargin=2pt,
leftmargin=0cm,
rightmargin=0cm,
linewidth=2pt,
splitbottomskip=0pt]{theoremBox}


%Cornice definizione
\newmdenv[skipabove=5pt,
skipbelow=5pt,
rightline=false,
leftline=true,
topline=false,
bottomline=false,
linecolor=blue,
backgroundcolor=blue!0,
innerleftmargin=5pt,
innerrightmargin=5pt,
innertopmargin=0pt,
leftmargin=0cm,
rightmargin=0cm,
linewidth=2pt,
innerbottommargin=2pt]{definitionBox}

%Cornice proposizione
\newmdenv[skipabove=5pt,
skipbelow=5pt,
rightline=false,
leftline=true,
topline=false,
bottomline=false,
linecolor=teal,
backgroundcolor=teal!0,
innerleftmargin=5pt,
innerrightmargin=5pt,
innertopmargin=0pt,
leftmargin=0cm,
rightmargin=0cm,
linewidth=2pt,
innerbottommargin=2pt]{propositionBox}

%Cornice lemma
\newmdenv[skipabove=5pt,
skipbelow=5pt,
rightline=false,
leftline=true,
topline=false,
bottomline=false,
linecolor=orange,
backgroundcolor=orange!0,
innerleftmargin=5pt,
innerrightmargin=5pt,
innertopmargin=0pt,
leftmargin=0cm,
rightmargin=0cm,
linewidth=2pt,
innerbottommargin=2pt]{lemmaBox}

%Cornice corollario
\newmdenv[skipabove=5pt,
skipbelow=5pt,
rightline=false,
leftline=true,
topline=false,
bottomline=false,
linecolor=green,
backgroundcolor=green!0,
innerleftmargin=5pt,
innerrightmargin=5pt,
innertopmargin=0pt,
leftmargin=0cm,
rightmargin=0cm,
linewidth=2pt,
innerbottommargin=2pt]{corollaryBox}


% Definizione degli ambienti con cornici senza cambiare i nomi
\let\oldtheorem\theorem
\renewenvironment{theorem}{\begin{theoremBox}\oldtheorem}{\end{theoremBox}}
\let\oldlemma\lemma
\renewenvironment{lemma}{\begin{lemmaBox}\oldlemma}{\end{lemmaBox}}
\let\oldcorollary\corollary
\renewenvironment{corollary}{\begin{corollaryBox}\oldcorollary}{\end{corollaryBox}}
\let\oldproposition\proposition
\renewenvironment{proposition}{\begin{propositionBox}\oldproposition}{\end{propositionBox}}
\let\olddefinition\definition
\renewenvironment{definition}{\begin{definitionBox}\olddefinition}{\end{definitionBox}}

% Ambiente per osservazioni
\newenvironment{oss}
{\par\indent\small{\textbf{Osservazione}}}
{\par}

\newcommand{\subtitle}[1]{%
  \posttitle{%
    \par\end{center}
    \begin{center}\large#1\end{center}
    \vskip0.5em}%
}
%Scorciatoie:
\newcommand{\R}{\mathbb{R}}
\newcommand{\U}{\mathcal{U}}