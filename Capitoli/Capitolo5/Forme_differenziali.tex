\chapter{Forme differenziali}
Si passi ora alla trattazione delle forme differenziali.
Per fare ciò, si consideri $A \subseteq \mathbb{R}^n$ aperto. Siano poi $a_1(x), \dots, a_n(x)$ con $x \in A$ funzioni continue in $A$ e siano $dx_1, \dots, dx_n$ funzioni lineari da $\R^n$ a $\R$ tali che $h \mapsto h_i$.
\begin{definition} \label{Def: Forma differenziale}
Si dice \textbf{forma differenziale} un'applicazione $\omega:\ x \in A \to  (\mathbb{R}^n)^*$, cioè il duale di $\mathbb{R}^n$, data dall'espressione:
\begin{equation}
    \omega= \omega(x)= \sum\limits_{i=1}^{n}{a_i(x)dx_i}
\end{equation}
con $a_i$ detti \textbf{coefficienti} della forma differenziale $\omega$.
\end{definition}
\begin{oss}
    Se $h$ è un vettore generico di $\R^n$, il valore assunto da $\omega(x)$ in $h$ è dato da:
    \begin{equation}
        \omega(x)(h)=\sum\limits_{i=1}^{n}{a_i(x)dx_i(h)}= \sum\limits_{i=1}^{n}{a_i(x)h_i}
    \end{equation}
\end{oss}
\begin{oss}
A tal proposito, si dice \textit{campo vettoriale} associato la funzione $F: A\subseteq \R^n \to \R^n$ data da
\begin{equation}
    F=F(x)=(a_1(x), \dots, a_n(x_n))
\end{equation}
Inoltre, si può notare che il legame tra i due è dato da:
\begin{equation} \label{Eq: Campo vettoriale}
    \omega(x)= \langle F(x), dx \rangle
\end{equation}
dove $dx= (dx_1,\dots, dx_n)$.
\end{oss}
\begin{oss}
Presa $f$ differenziabile in $A$, si può osservare che il suo differenziale $df$ sarà dato da
\begin{equation} \label{Eq: Differenziale}
    df(x)= \sum\limits_{i=1}^{n}{\frac{\partial f}{\partial x_i}(x)\ dx_i}= \langle \nabla f(x), dx\rangle
\end{equation}
\end{oss}
\section{Integrale curvilineo di seconda specie}
\begin{definition} \label{Def: Integrale curvilineo di seconda specie}
    Sia $A \subseteq \R$ e sia $\omega: A \to (\R)^*$ con coefficienti continui. Sia inoltre $\gamma$ una curva regolare di parametrizzazione $\varphi: [a, b] \to A$. Allora si definisce \textbf{integrale curvilineo di seconda specie} o integrale di $\omega$ lungo $\gamma$ la quantità:
    \begin{equation} \label{Eq: Integrale curvilineo di seconda specie}
        \int\limits_\gamma \omega := \int\limits_{a}^{b}\left( \sum\limits_{i=1}^{n}{a_i(\varphi(t))\ \varphi_i'(t)}\right)\, dt
    \end{equation}
\end{definition}
Per tale ragione, si può osservare che l'integrale curvilineo di seconda specie di un campo vettoriale $F$ associato ad una forma differenziale $\omega$ lungo una curva $\gamma$ è dato dall'integrale curvilineo di prima specie di 
\begin{equation}
    \int\limits_{\gamma}{\langle F, T \rangle}\, ds
\end{equation}
e in particolare vale:
\begin{equation}
    \int\limits_{\gamma}{\omega}= \int\limits_{a}^{b} {\langle F(\gamma(t)), \gamma'(t)\rangle}\, dt = \int\limits_{a}^{b} {\langle F(\gamma(t)), T(\gamma(t))\rangle} |\gamma'(t)|\, dt = \int\limits_{\gamma}{\langle F, T \rangle}\, ds
\end{equation}
Inoltre, vale il seguente risultato.
\begin{theorem}[Invarianza per equivalenza di curve] \label{Teo: Invarianza per equivalenza di curve dell'integrale curv di 2 specie}
    Siano $\gamma,\ , \gamma'$ due curve equivalenti e sia $\omega$ una forma differenziale definita in $A$. Allora
    \begin{equation}
        \int\limits_\gamma { \omega} = \int\limits_{\gamma'}{\omega}\quad \text{oppure} \quad \int\limits_\gamma { \omega} = -\int\limits_{\gamma'}{\omega}
    \end{equation}
    se sono percorse nello stesso verso o in verso opposto, rispettivamente.
\end{theorem}
\begin{proof}
    Per definizione di equivalenza di curve, prese 
    $\varphi: [a, b] \to \R^n$ parametrizzazione di $\gamma$ e $\psi:[c,d] \to \R^n$ parametrizzazione di $\gamma'$,  esiste un cambio di parametro ammissibile $\eta$ tale che $\varphi(t)= \psi(\eta(t))$ per ogni $t \in [a,b]$. Allora, 
    \begin{equation}
        \int\limits_\gamma {\omega}= \int\limits_{a}^{b}{\langle F(\varphi(t)), \varphi'(t) \rangle}\, dt = \int\limits_{a}^{b}{ \langle F(\psi(\eta(t))), \psi(\eta(t))\eta'(t)\rangle}\, dt
    \end{equation}
    dove $F$ è il campo vettoriale associato ad $\omega$.
    Perciò, mediante la sostituzione
    \begin{equation}
        \eta(t)=s \qquad \eta'(t)dt =ds
    \end{equation}
    si ha che, se le due curve sono percorse nello stesso verso, $\eta'(t) >0 $ e $\eta(a)=c,\ \eta(b)=d$ e
    \begin{equation}
    \int\limits_{\eta(a)=c}^{\eta(b)=d}{ \langle F(\psi(s)), \psi'(s)\rangle}\, ds \overset{\text{Costr.}}{=} \int\limits_{\gamma'}{\omega} 
    \end{equation}
    Altrimenti, se $\eta'(t) < 0 $,  $\eta(a)=d,\ \eta(b)=c$ e 
    \begin{equation}
     \int\limits_{\eta(a)=d}^{\eta(b)=c}{ \langle F(\psi(s)), \psi'(s)\rangle}\, ds = - \int\limits_{c}^{d}{\langle F(\psi(s)), \psi'(s)\rangle}\, ds\overset{\text{Costr.}}{=} -\int\limits_{\gamma'}{\omega} 
    \end{equation}
\end{proof}
\begin{oss}
    Per come è definito, si può notare che $\int\limits_{\gamma}{\omega}$ è il lavoro del campo vettoriale $F$ lungo $\gamma$.
\end{oss}
\section{Forme differenziali esatte}
In continuità con l'ultima osservazione, può essere approfondito il tema del lavoro di una forma differenziale lungo una certa curva $\gamma$. In particolare, scopo del paragrafo è stabilire sotto quali circostanze una forma differenziale possa dirsi \textit{esatta} e come tale gruppo possa essere caratterizzato.
\begin{definition} \label{Def: Forma differenziale esatta}
    Sia $\omega$ una forma differenziale su $A \subseteq \R^n$. Allora $\omega$ è detta \textbf{esatta} in $A$ se ammette primitiva $f$, cioè se essa è il differenziale di una certa $f \in C^1(A)$ tale che 
    \begin{equation}
        \omega = df
    \end{equation}
    In maniera equivalente, una forma differenziale esatta è una forma differenziale tale che
    \begin{equation} \label{Eq: Forma differenziale esatta}
        \exists\ \mathcal{U} \in C^1(A) \ \text{tale che}\ F=\nabla\mathcal{U}
    \end{equation}
\end{definition}
In termini fisici, così come prima si definiva campo vettoriale la funzione $F$ associata a $\omega$, ora, si definisce \textit{potenziale} $\U$ ciò che in analisi matematica è detto primitiva. Inoltre, sempre nell'ambito del lessico fisico, si può creare un ulteriore ponte sottolineando che una forma differenziale esatta è talvolta chiamata \textit{campo vettoriale conservativo}. Infatti, si può dimostrare che il lavoro di una forma differenziale esatta è dato dalla differenza di primitive valutate all'istante iniziale e a quello finale.
\begin{theorem} \label{Teo: Integrale di forme differenziali esatte}
    Sia $\omega$ una forma differenziale esatta e $\varphi: [a, b] \to \R^n$ una parametrizzazione della curva $\gamma$ regolare. Allora, detta $f$ primitiva di $\omega$, si ha che 
    \begin{equation}
        \int\limits_\gamma{\omega}= f(\varphi(b))-f(\varphi(a))
    \end{equation}
\end{theorem}
\begin{proof}
    Si risolva tale integrale.
    \begin{equation}
    \begin{aligned}
        \int\limits_{\gamma}{\omega} &\overset{\text{Def}}{=}\int\limits_{a}^{b}{\langle F(\varphi(t)), \varphi'(t)\rangle}\, dt \overset{\eqref{Eq: Forma differenziale esatta}}{=} \int\limits_{a}^{b}{\langle \nabla f(\varphi(t)), \varphi'(t) \rangle}\, dt =\\
        &\overset{\eqref{Eq: Derivata composta 1}}{=}\int\limits_{a}^{b}\frac{d}{dt}{f(\varphi(t))} \overset{\text{TFC}}{=} f(\varphi(b))- f(\varphi(a))
    \end{aligned}
    \end{equation}
\end{proof}
\paragraph{Giustificazione del termine \textit{conservativo}}
Si pensi ad $F$ come un campo di forze che agisce su una massa di $m=1$ e si consideri una generica traiettoria data da una curva $x=x(t)$ almeno di classe $C^2$ che soddisfi l'equazione differenziale 
\begin{equation}
    x''(t)=F(x(t)) \qquad \forall\ t \in \R
\end{equation}
Si introduca poi l'energia meccanica 
\begin{equation}
    E_m(t)= \frac{1}{2} |x'(t)|^2- f(x(t))
\end{equation}
con $f$ potenziale di $F$.\\
Allora, derivando tale relazione, si ha che
\begin{equation}
\begin{aligned}
        \frac{d}{dt}{E_m(t)}&=\frac{1}{2} \frac{d}{dt}{ \langle x'(t), x'(t) \rangle}- \frac{d}{dt}{f(x(t))}=\\
        &= \langle x''(t), x'(t) \rangle- \langle \nabla f(x(t)), x'(t) \rangle =\\
        &= \langle F(x(t)), x'(t) \rangle - \langle F(x(t)), x'(t) \rangle =0
\end{aligned}    
\end{equation}

Ciò significa che l'energia meccanica si \textit{conserva} lungo le traiettorie indotte dal campo $F$.
\begin{theorem}[Caratterizzazione delle forme esatte] \label{Teo: Caratterizzazione forme esatte}
Siano $A \subseteq \R^n$ aperto connesso, $\omega$ una forma differenziale in $A$ e $\gamma,\ \gamma_1,\ \gamma_2$ tre curve regolari a tratti con sostegno contenuto in $A$. Allora le seguenti affermazioni sono equivalenti:
\begin{enumerate}
    \item $\omega$ è esatta in A
    \item se $\gamma$ è chiusa allora 
    \begin{equation}
        \int\limits_{\gamma}{\omega}=0
    \end{equation}
    \item se $\gamma_1$, $\gamma_2$ hanno gli stessi estremi e lo stesso verso di percorrenza allora
    \begin{equation}
        \int\limits_{\gamma_1}{\omega} = \int\limits_{\gamma_2}{\omega}
    \end{equation}
\end{enumerate}
\end{theorem}
\begin{proof}
    Si dimostrino le tre affermazioni in questo ordine: $1 \Rightarrow 2 \Rightarrow 3 \Rightarrow 1$.\\
    $1 \Rightarrow 2$: Sia $\varphi: [a,b] \to A$ una parametrizzazione di $\gamma$. Poiché $\gamma$ è chiusa, si ha che $\varphi(a)=\varphi(b)$. Allora, siccome $\omega$ è esatta, vale 
    \begin{equation}
        \int\limits_{\gamma}{\omega} = f(\varphi(b))- f(\varphi(a)) = f(\varphi(a))-f(\varphi(a))=0
    \end{equation}
    $2 \Rightarrow 3$: Si considerino $\gamma_1,\ \gamma_2$ definite come nelle ipotesi di $(3)$. Siano poi:
    \begin{equation}
        \begin{aligned}
            &\varphi: [a,b] \to A \ \text{parametrizzazione di } \gamma_1\\
            &\psi:[c, d] \to A \ \text{parametrizzazione di } \gamma_2
        \end{aligned}
    \end{equation}
    Sia poi $\hat{\psi}$ la parametrizzazione che percorre $\psi$ in verso opposto, cioè
    \begin{equation}
        \hat{\psi}(t)= \psi(-t+b+d) \qquad t \in [b, b+d-c]
    \end{equation}
Si può verificare che: $\hat{\psi}(b)= \psi(d)$ e $\hat{\psi}(b+d-c)=\psi(c)$.\\
Allora, definita $\xi: [a, b+d-c] \to A$ come
\begin{equation}
    \xi(t)= \begin{cases}
        \varphi(t) \qquad & t \in [a,b]\\
        \hat{\psi}(t) \qquad & t \in [b, b+d-c]
    \end{cases}
\end{equation}
Si ha che per costruzione essa è regolare a tratti. Inoltre, si ha che:
\begin{equation}
\begin{aligned}
    &\xi(a)= \varphi(a)\\
    &\xi(b+d-c)= \hat(\psi(b+d-c))=\psi(c)=\varphi(a)
\end{aligned}
\end{equation}
cioè $\xi(t)$ è una curva chiusa. Allora, detta $\hat{\gamma}$ la curva di parametrizzazione $\xi(t)$, per il punto (2) si ha che:
\begin{equation}
\begin{aligned}
\int\limits_{\hat{{\gamma}}}{\omega} &= \int\limits_{a}^{b} \langle F(\varphi(t)), \varphi'(t)\rangle \, dt + \int\limits_{b}^{b+d-c}{\langle F(\hat{\psi}(t), \hat{\psi}(t) \rangle}\, dt=\\ 
&= \int\limits_{a}^{b} \langle F(\varphi(t)), \varphi'(t)\rangle \, dt - \int\limits_{c}^{d}{\langle F(\psi(t)), \psi'(t) \rangle}\, dt =\\
&= \int\limits_{\gamma_1}{\omega} - \int\limits_{\gamma_2}{\omega} = 0
\end{aligned}
\end{equation}
Da cui si ottiene la tesi.\\
$3 \Rightarrow 1$: Sia $x_0 \in A$ fissato. Dato $x \in A$, poiché $A$ è connesso per archi, esiste una curva $\gamma$ regolare a tratti che congiunga $x_0$ a $x$ con sostegno in A. Sia inoltre $\gamma$ percorsa da $x_0$ a $x$.\\
Si definisca allora
\begin{equation}
    f(x):=\int\limits_{\gamma}{\omega}
\end{equation}
In particolare si può osservare che grazie a (3) tale funzione è sempre ben definita e il suo valore non dipende dalla curva scelta.\\
L'obiettivo della dimostrazione è mostrare che $f \in C^1(A)$ e $df= \omega$.\\
Siano dunque $x=(x_1, \dots, x_n) \in A$, $h \in \R$ tali che $x+he_i \in A$. Sia poi $\varphi$ la curva di sostegno $[x, x+he_i]$, cioè
\begin{equation}
    \varphi= \begin{cases}
        x_1(t)=x_1\\
        \vdots\\
        x_i(t)=x_i+ht\\
        \vdots\\
        x_n(t)=x_n
    \end{cases}
    \qquad t \in [0,1]
\end{equation}
Sia poi $\gamma_1= \gamma \cup \varphi$, che, per costruzione, è regolare a tratti e congiunge $x_0$ a $x_0+he_i$. Tramite la definizione di $f$ si ottiene che
\begin{equation}
\begin{aligned}
    f(x_0+he_i)-f(x)&=\int\limits_{\gamma_1}{\omega}- \int\limits_{\gamma}{\omega}= \int\limits_{\gamma}{\omega}+\int\limits_{\varphi}{\omega}-\int\limits_{\gamma}{\omega}= \int\limits_{\varphi}{\omega}=\\
    &=\int\limits_{0}^{h}{\left[\sum\limits_{j=1}^{n}{a_j(\varphi(t))\varphi_j'(t)} \right] }\, dt=    \int\limits_{0}^{h}{a_i(\varphi(t))\cdot 1}\, dt =\\ 
    &=\int_{0}^{h}{a_i(x_1, \dots, x_i+ht, \dots x_n)}\, dt
\end{aligned}
\end{equation}
Di conseguenza, 
\begin{equation}
\begin{aligned}
    \frac{\partial{f}}{\partial x_i}{(x)}&= \lim_{h\to 0}{\frac{f(x+he_i)-f(x)}{h}}= \frac{1}{h}{\int_{0}^{h}{a_i(x_1, \dots, x_i+ht, \dots x_n)}\, dt}=\\
    &\overset{\text{TFC}}{=} a_i(x_1, \dots, x_i, \dots, x_n) = a_i (x)
\end{aligned}
\end{equation}
Reiterando per ogni $i$ si ottiene che $\nabla f(x)= (a_1(x), \dots, a_n(x))$, cioè $\omega$ è una forma esatta. 
\end{proof}
%\newpage
%\section{Forme differenziali chiuse}
%\section{Formule di Gauss-Green}